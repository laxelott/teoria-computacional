\documentclass[12pt,letterpaper]{article}
\usepackage[latin1]{inputenc}
\usepackage{amsmath}
\usepackage{amsfonts}
\usepackage{amssymb}
\usepackage{graphicx}
\usepackage{multicol}
\usepackage{changepage}
\usepackage{float}
\usepackage{cite}
\usepackage{url}


\usepackage{listings}
\usepackage{xcolor}

\lstdefinestyle{sharpc}{language=[Sharp]C, frame=lr, rulecolor=\color{blue!80!black}}

\lstset{
    showstringspaces=false, % don't mark spaces in strings
    numbers=left, % display line numbers on the left
    commentstyle=\color{blue}, % comment color
    keywordstyle=\color{magenta}, % keyword color
    stringstyle=\color{red} % string color
}

\usepackage[left=2cm,right=2cm,top=2cm,bottom=2cm]{geometry}
\author{Santiago P�rez Carlos Augusto}

%Portada
\begin{document}
	\pagestyle{plain}{
		\pagestyle{empty}
		\noindent
		{\small
		\begin{tabular}{
			p{0.75\textwidth} p{0.25\textwidth}}
			\includegraphics[scale=0.07]{../Imagenes/ipn.png}  &  
			\includegraphics[scale=0.35]{../Imagenes/Escom.png} 
		\end{tabular}
		}
		\begin{center}
			\par\vspace{2cm} 
			{
				\Huge\textbf{
				Instituto Polit�cnico Nacional \\*[0.20cm] Escuela Superior de C�mputo,
				}
			}
			\par\vspace{0.5cm}
			{
				\Large\textbf{
				Ingenier�a en sistemas computacionales \\ Materia: Teor�a computacional \\ Profesor: Ju�rez Mart�nez Genaro \\ Grupo: 2CM5
				}
			}
			\par\vspace{1cm}
			{
				\large\textbf{Pr�cticas 3er parcial } 
				
				\large\textbf{Pr�ctica 1} 
				\large\textbf{Aut�mata Pila} 
			}
			\par\vspace{1cm}
			{
				\large\textbf{Trevi�o Palacios Axel \\
					20/01/2020 } 
			}
			\par\vspace{3cm}
			
		\end{center}
		\clearpage

	}
%Documento
	
	\section*{Practica 1}
	El siguiente programa realiza el funcionamiento de un aut�mata de pila, en el cual, cada vez que vaya recorriendo mis estados hasta uno v�ilod, debe de ir vaciando la pila.
	\subsection*{Objetivo}
	El objetivo de la pr�ctica mostrar el funcionamiento del aut�mata de pila, ingresando el valor cada vez que se cumple un 0 e ir sacando (vaciando) la pila cada vez que en los estados siguientes sea un 1, para que se pueda aceptar la cadena.
	\subsection*{C�digo}
		\lstset{style=sharpc}
	
	\begin{lstlisting}
%Aqui va el codigo
	\end{lstlisting}
	
	\subsection*{Capturas de pantalla}
	\begin{center}
	
	
	\end{center}		
	
	\section*{Conclusiones}
Al realizar la pr�ctica se analiz� la forma en la cual se pueden almacenar cada nodo identificando si es un estado final, estado inicial, o el estado en la cual la transici�n es 1, para que se pueda eliminar de la pila el proceso, entonces cada vez que no se termine en 01 no se cumple el estado final, por lo cual la cadena no va a ser recibida.

\end{document}